\documentclass{article}
\usepackage{graphicx} % Required for inserting images
\usepackage{hyperref}
\usepackage{indentfirst}

\title{TOB - Projet long : sujets}
\author{Groupe GH3 : \\ Yassine Bougacha, Leïlie Canillac, Noé Engel, Baptiste Gomez, \\ Vianney Hervy, Amélie Rouge, Sabatier Thomas}
\date{12 Février 2024}

\begin{document}

\maketitle
\newpage

\tableofcontents
\newpage

\section{Simulateur de vie particulaire}

Fondamentalement, l'univers est juste composé de particules qui interagissent. Ce projet propose une modélisation d'un ensemble de particules ponctuelles de propriétés différentes. Ces particules sont divisées en plusieurs familles. Chaque particule d'une même famille comporte les mêmes caractéristiques : couleur, masse, relation aux particules étrangères. Ces relations peuvent pousser deux particules non-soeurs à se repousser ou s'attirer l'une l'autre, voire se stabiliser à une distance fixée. En augmentant le nombre de famille et en choisissant les bons paramètres, qui sait ce que ces particules peuvent devenir ? \bigskip

Ce modèle s'inspire à la fois de l'infiniment grand : les objets célestes soumis aux forces de gravitation, et de l'infiniment petit : les micro-organismes, cellules, molécules ou atomes qui composent la matière. \bigskip

Une telle application, en plus de générer des animations colorées mesmérisantes, produit des situations d'interactions intéressantes. \bigskip

L'utilisateur sera libre du nombre et des différentes propriétés des familles de particules, notamment les différentes interactions entre les familles, que ce soit à des fins de simulation ou simplement esthétiques. \bigskip

La dimension imprévisible des conséquences des interactions entres particules peut mener à la création d'entités plus grosses (système solaire, troupeaux) : des interactions locales engendrent des interactions globales. \bigskip

Aucune dimension d'évolution ou de sélection naturelle n'est prévue, ce n'est pas un simulateur d'évolution. \bigskip

Ce projet peut potentiellement servir dans un cadre éducatif, d'analyse et récréatif. Il permettrait à tous de comprendre comment des systèmes complexes tels que le cerveau humain émergent d'éléments et de règles simples. \bigskip

\newpage

\section{Jeu roguelike}

Le roguelike est un sous-genre de jeu vidéo de rôle dans lequel le joueur explore un donjon infesté de monstres qu'il doit combattre pour gagner de l'expérience et des trésors. Le genre se caractérise notamment par la génération procédurale de ses niveaux et par son système de mort permanente (la partie est finie dès la mort du joueur). \bigskip

L'objectif principal de notre projet est de concevoir un jeu roguelike où chaque partie est une nouvelle aventure, aucune partie ne se ressemble et c'est là la plus grande force de ce type de jeu qui date des années 80. \bigskip

La présence d'un inventaire ainsi que d'items (trésors, armes etc) trouvable lors d'une partie permettent une infinité de possibilités et de stratégies pour le joueur qui est ainsi poussé à l'innovation et à expérimenter toujours plus. \bigskip

L'exploration étant un élément important, chaque environnement sera créé aléatoirement, garantissant un intérêt constant des joueurs ainsi qu'une diversité dans les aventures vécues. \bigskip

Dans la même direction que la génération aléatoire de donjons, les ennemis rencontrés pourront être dotés de capacités spéciales dont le joueur aura la surprise. Il aura alors peu de temps pour réagir et trouver une manière de contrer celles-ci, usant notamment des objets de son inventaire. \bigskip

La présence d'un système de mort permanente renforce l'engagement du joueur et permet de l'immerger plus dans le jeu ainsi que de lui faire ressentir de fortes émotions, par exemple si son personnage est au bord de la mort. De plus, ce concept permet une gratification de la progression du joueur : l'objectif n'est pas de tout réussir dès la première partie mais bien de vivre un challenge, de ressentir le besoin de se dépasser et enfin la gratification d'avoir enfin réussi une partie. \bigskip

Un paramètre de difficulté permettra aux joueurs débutants et avancés de prendre du plaisir et de jouer à leur rythme. Ce paramètre influencera le choix des profils d'ennemis, l'architecture des donjons et les attributs/pouvoirs des items.

\newpage

\section{Simulation d'un écosystème virtuel}

Un écosystème est un ensemble d'êtres vivants qui interagissent (par exemple une relation prédateur - proie) et qui évolue au fil du temps. Notre objectif est de créer un cadre qui permette à n'importe qui de créer ses propres simulations, avec différents prédateurs, proies, ressources et d'observer l'évolution de son écosystème dans le temps. \bigskip

Ce concept de modélisation flexible permet de constater les conséquences de divers changements dans un écosystème. \bigskip

En effet, cette modélisation permettra d'observer l'impact de différents paramètres sur l'évolution d'un écosystème. Il sera possible de choisir l'environnement de l'écosystème, qui déterminera les conditions climatiques et la quantité de ressources disponibles, les espèces présentes dans notre écosystème, chaque espèce ayant ses propres caractéristiques impactant sur sa survie et son taux de reproduction. \bigskip

On pourra aussi définir un taux de mutation qui fera muter aléatoirement les caractéristiques de certaines espèces. \bigskip

Le but de ces simulations étant de proposer à l'utilisateur une visualisation facilement compréhensible de l'évolution de l'écosystème créé, en lui permettant de faire varier les paramètres de simulations et de créer ses propres espèces et les interactions qui en résultent. \bigskip

Le projet pourrait ainsi inclure des outils d'analyse et de visualisation de données permettant aux utilisateurs de visualiser les tendances, générer des graphiques. \bigskip

Ce projet peut potentiellement servir dans un cadre éducatif, d'analyse et récréatif.

\newpage

\section{Application de gestion de son empreinte carbone}

Dans le cadre de la transition écologique, chacun est concerné et responsable de son impact. Malgré tout, la motivation ne suffit pas et il est souvent difficile de faire un suivi de ses émissions carbone. C'est pourquoi nous avons pensé à cet outil. Cette application permet de prévoir et de suivre son empreinte carbone en temps réel. Elle propose les fonctionnalités suivantes :

\begin{itemize}
    \item \textbf{Suivi personnalisé des émissions de carbone :}  vous pouvez fournir les détails de vos trajets en voiture, en avion ou autres, vos listes d'achats (nourriture, vêtements etc) et vos autres consommations.
    \item \textbf{Analyse détaillée des habitudes de consommation :} en analysant vos données, notre application identifie les domaines où vous émettez le plus de carbone pour éclairer vos choix futurs : dans quel domaine êtes-vous un bon élève ?
    \item \textbf{Conseils personnalisés :} les deux champs précédents nourriront notre algorithme de conseil qui vous aidera à améliorer votre bilan carbone. Il vous proposera des alternatives plus écologiques à vos produits achetés par exemple.
    \item \textbf{Suivi des progrès et objectifs :} en continuité avec la seconde fonctionnalité, notre application vous présentera vos statistiques au cours du temps. Vous pourrez vous fixer des objectifs et l'application vous proposera des moyens pour les atteindre. En accomplissant vos objectifs vous pourrez gagner des badges.
    \item \textbf{Défis au sein de l'application :} On vous proposerait des défis hebdomadaires qui pourraient vous permettre de vous challenger et de vous encourager à réduire les émissions de carbone, avec des récompenses à la clé.
    \item \textbf{Aide en direct :} Pour vous aider à faire vos achats dans la vie quotidienne, il serait possible de scanner deux produits, et de vous aider à choisir le plus écologique. Pour un itinéraire précis, il serait possible de choisir le meilleur mode de transport ou bien la meilleure route à emprunter, pour pouvoir minimiser vos empreinte carbone.
    \item \textbf{Partage :} À la manière d'autres applications de progression personnelles telles que Strava (sport), notre application propose de résumer vos statistiques sous forme d'images infographiques que vous pourrez alors partager sur vos réseaux préférés.
\end{itemize}

Cette application est à destination de particuliers mais pourrait être l'objet d'une mise-à-jour la rendant accessible aux entreprises désireuses d'améliorer leur empreinte carbone.


\end{document}



